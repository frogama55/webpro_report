\chapter{千葉工業大学情報変革科学部情報工学科1年次の授業一覧\\システムの開発者向け仕様書}
\section{データ構造}
\subsection{変数定義}
本システムでは,データベースソフトウェアを使用せず,アプリケーションサーバ(IDapp.js)内の配列変数jugyouにデータを格納する方法を採用している.jugyouはオブジェクトの配列変数であり,プロセス終了時にデータは消失する.

\subsection{オブジェクトの構造}
jugyou配列の各要素は,以下のプロパティを持つオブジェクトである.変数名と実データの対応関係は表\ref{KANKEI_jugyou}の通りである\cite{sankou_jugyou}.


\begin{table}[H]
    \centering
    \caption{変数名と実データの対応関係}
    \label{KANKEI_jugyou}
    \begin{tabular}{|c|c|c|c|}
        \hline
        変数名 & データ型 & 本システムでの役割 & データ例  \\ \hline
        id & Number & データの管理ID & 1   \\ \hline 
        code & String & 分野名 & "教養基礎科目"  \\ \hline
        name & String & 授業名 & "英語理解基礎1"  \\ \hline
        change & String & 必修か選択か & "選択"  \\ \hline
        passengers & Number/String & 単位数 & 1  \\ \hline
        distance & Number/String & 備考 & "基礎レベル対象の科目" \\ \hline
    \end{tabular}
\end{table}

\newpage
\section{HTTPメソッドとリソース名}
表\ref{KANKEI_jugyou_HTTP}に,リソース名とHTTPメソッドを定義する.
\begin{table}[H]
    \centering
    \caption{変数名と実データの対応関係}
    \label{KANKEI_jugyou_HTTP}
    \small
    \begin{tabularx}{\textwidth}{|c|c|l|X|}
        \hline
        機能 & HTTPメソッド & \multicolumn{1}{c|}{リソース名(URL)} & \multicolumn{1}{c|}{処理内容}  \\ \hline
        一覧表示 & GET & /jugyou & データ一覧画面(jugyou.ejs)を表示する.   \\ \hline 
        詳細表示 & GET & /jugyou/:number & 指定IDの詳細画面(jugyou\_detail.ejs)を表示する.   \\ \hline
        新規登録(画面) & GET & /jugyou/create & 新規登録フォーム(jugyou.html)へリダイレクトする.   \\ \hline
        新規登録(処理) & POST &/jugyou & フォームからデータを受け取り,配列に追加する.  \\ \hline
        編集(画面) & GET & /jugyou/edit/:number& 編集フォーム(jugyou\_edit.ejs)を表示する.   \\ \hline
        編集(処理) & POST & /jugyou/update/:number & 既存データを更新し,一覧表示画面へリダイレクトする. \\ \hline
        削除(画面) & GET & /jugyou/delete/confirm/:number & 削除確認画面(jugyou\_delete.ejs)を表示する.\\ \hline
        削除(処理) & POST &/jugyou/delete/:number & データをnull化し,一覧表示画面へリダイレクトする. \\ \hline
    \end{tabularx}
\end{table}

\newpage
\section{ページ遷移}
ページ遷移の様子を,図\ref{jugyou_page}に示す.
\begin{figure}[H]
\begin{center}
\includegraphics[width=0.95\textwidth,clip]{fig/developer/jugyou_page.pdf}
\end{center}
\caption{\textgt{
    ページ遷移図
}}
\label{jugyou_page}
\end{figure}

\newpage
\section{機能詳細と実装上の注意点}
\subsection{新規データ追加機能(Create)}
この機能では,jugyou配列の末尾に新しいオブジェクトを追加(push)し,「jugyou.length + 1」(一番最後のIDを1大きくしたもの)を新しいIDとして付与する.追加完了後,jugyou.ejsを用いて登録済みデータ一覧ページを表示する.

\subsection{一覧表示機能(Read)}
この機能では,EJSテンプレート内でjugyou配列をループ処理し,表形式で表示する.注意点として,削除処理によって,配列内にnullが含まれる可能性があるため,EJS側でif文等を用いた確認を行い,null要素を表示しないように制御する必要がある.

\subsection{更新機能(Update)}
この機能では,URLパラメータである:numberを用いて配列にアクセスし,フォームから送信された値(req.body)で詳細を上書きする.処理完了後は,res.redirect(`/jugyou')により,一覧画面へ遷移する. 

\subsection{削除機能(Delete)}
配列の要素を削除するために,該当部分にnullを代入する.これは,spliceを使用すると,削除した部分にあとのデータが詰められるため,既存データを指す:numberがずれてしまい,詳細表示や編集時のリンク先が誤ったデータを指すことがあったためである.ejs側では表記上削除されるようにしている.