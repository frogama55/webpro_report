\chapter{利用者向け仕様書}

\section{はじめに}
本システムは,「千葉工業大学の各学科の偏差値」,「千葉工業大学情報変革科学部情報工学科1年次の授業一覧」,「千葉工業大学情報変革科学部情報工学科の研究室」を一覧表示するWebシステムである.本仕様書では,代表として,「千葉工業大学情報変革科学部情報工学科の研究室」の利用方法について説明する.
なお,このシステムはパソコンでの利用を想定しているため,本仕様書ではパソコン上での動作等について説明する.

\newpage
\section{システムへのアクセス方法}
まず,Webブラウザを立ち上げる.ここでは,図\ref{icon_chrome}がアイコンであるGoogle chromeを利用する.

\begin{figure}[H]
\begin{center}
\includegraphics[width=0.2\textwidth,clip]{fig/user/chrome.png}
\end{center}
\caption{\textgt{
    Google chromeのアイコン.このアイコンをクリックして起動する.
}}
\label{icon_chrome}
\end{figure}

次に,図\ref{adressbar}のようなアドレスバーに<\url{http://localhost:8080}>を入力し,アクセス(パソコンのEnterキーを押下)する.

\begin{figure}[H]
\begin{center}
\includegraphics[width=0.75\textwidth,clip]{fig/user/adressbar.png}
\end{center}
\caption{\textgt{
    ブラウザの画面.赤枠部分がアドレスバー.
}}
\label{adressbar}
\end{figure}

各システムが並ぶトップページが表示されるので,「千葉工業大学情報工学科の研究室一覧」をクリックする.これにより,システムへアクセスできた.

\newpage
\section{システムの機能と操作方法}
\subsection{研究室情報の一覧}
\subsubsection{研究室情報の一覧の閲覧}
システムへアクセスすると,現在登録されている研究室の一覧が表示される.各行には,データに固有のID,研究室名,研究概要が表示されている.
「トップページへ戻る」をクリックすると,各システムの一覧画面に遷移する.
研究概要をクリックすると,各研究室の詳細画面にアクセスすることができる.図\ref{kenkyu_top}に研究室の一覧画面,図\ref{kenkyu_detail}に研究室情報の詳細画面をそれぞれ示す.

\begin{figure}[H]
\begin{center}
\includegraphics[width=0.75\textwidth,clip]{fig/user/kenkyu_top.png}
\end{center}
\caption{\textgt{
    研究室一覧画面.
}}
\label{kenkyu_top}
\end{figure}

\begin{figure}[H]
\begin{center}
\includegraphics[width=0.75\textwidth,clip]{fig/user/kenkyu_detail.png}
\end{center}
\caption{\textgt{
    研究室の詳細画面.ここではID2のページを例示.
}}
\label{kenkyu_detail}
\end{figure}

研究室の独自サイトがある場合に限り,図\ref{kenkyu_detail}のように「研究室サイトURL」の部分が青色になり,アクセス可能になる.クリックすると,新しいタブが開き,研究室の独自サイトに遷移する.
また,図\ref{kenkyu_detail}の下部にある「研究室一覧に戻る」をクリックすると,図\ref{kenkyu_top}に示した,研究室の一覧画面に戻ることができる.

\subsubsection{研究室情報の編集と削除}
図\ref{kenkyu_detail}の下部にある「編集」「削除」ボタンについて説明する.編集ボタンは,そのページに表示されている研究室データを好きなように編集することができる.クリックすると,図\ref{kenkyu_edit}のような編集画面に遷移する.

\begin{figure}[H]
\begin{center}
\includegraphics[width=0.75\textwidth,clip]{fig/user/kenkyu_edit.png}
\end{center}
\caption{\textgt{
    研究室情報の編集画面.ここではID2のページを例示.
}}
\label{kenkyu_edit}
\end{figure}

各項目のデータを書き換えることができる.書き換えたあと,下部にある「送信」ボタンをクリックすることで編集が完了する.また,「研究室一覧に戻る」をクリックすると,編集内容を保存せずに,図\ref{kenkyu_top}に示した,研究室の一覧画面に戻ることができる.


\subsection{研究室情報の追加}
図\ref{kenkyu_top}の下部にある「追加」ボタンについて説明する.このボタンをクリックすると,図\ref{kenkyu_add}のような追加画面に遷移する.各枠の右側にある項目の通りに,枠にデータを入力し,下部の「送信」ボタンをクリックすると,研究室一覧画面の一番下に,新たなIDとともにデータを追加することができる.

\begin{figure}[H]
\begin{center}
\includegraphics[width=0.75\textwidth,clip]{fig/user/kenkyu_add.png}
\end{center}
\caption{\textgt{
    研究室情報の追加画面.
}}
\label{kenkyu_add}
\end{figure}

\newpage
\section{その他}
このシステムの注意点として,システムの仕様上,ここで追加や編集,削除した内容は,サーバーを停止や再起動すると,初期化された状態に戻る.