\chapter{管理者向け仕様書}
\section{はじめに}
本システムは,Node.js上で動作するWebアプリケーションサーバであり,「千葉工業大学の各学科の偏差値(/hensachi)」,「千葉工業大学情報変革科学部情報工学科1年次の授業一覧(/jugyou)」,「千葉工業大学情報変革科学部情報工学科の研究室(kenkyu)」を一覧表示する3つのシステムを1つのサーバプロセスで稼働するものである.本システムは,HTTPリクエストを受け付け,EJSテンプレートエンジンを用いて動的にHTMLを用意し,クライアント(利用者)へ返す.


\section{システムの動作環境}
本システムは,パソコンでの利用を想定しているため,本仕様書ではパソコン上での動作等について説明する.開発環境がmacOSであるため,macOSでの説明を記述している.
本システムを稼働させるためには表\ref{kankyo}に示す環境が必要である.

\begin{table}[H]
    \centering
    \caption{動作環境要件一覧}
    \label{kankyo}
    \begin{tabular}{|c|c|}
        \hline
        各環境 & 必要な動作環境  \\ \hline
        実行するための環境 & Node.js   \\ \hline 
        パッケージ管理ツール & npm(Node.js内に同梱されている)  \\ \hline
        OS & Node.jsが動作するOS(Windows,macOS,Linux等)  \\ \hline
        ネットワーク & TCPポート8080番が使用可能であること  \\ \hline
    \end{tabular}
\end{table}

\newpage
\section{システムの導入手順}
\subsection{ソースコードの入手}
Githubリポジトリからソースコードをサーバー上の任意のディレクトリにクローン配置する.以下のコードをターミナルを用いて任意のディレクトリで実行することで可能である.

\verb|$git clone https://github.com/frogama55/webpro_report.git|

\verb|$cd webpro_report|

\subsection{使用パッケージのインストール}
本システムはexpressパッケージとejsパッケージを使用している.そのため,package.jsonが存在するディレクトリにて,ターミナルを用いて以下のコードを実行し,インストールする.

\verb|$npm install|

これにより,Node\_modulesというディレクトリが自動で作成され,パッケージが格納される.

\newpage
\section{システムの起動と停止}
\subsection{起動方法}
\label{kido}
IDapp.jsが存在するディレクトリにて,以下のコードを実行することで,サーバを起動することができる.

\verb|$npm start|

このとき,以下のメッセージが表示されれば,起動に成功している.

\verb|Example app listening on port 8080!|

\subsection{動作確認}
Webブラウザを起動し,以下のURLをアクセスする.使用するWebブラウザは,Safariではなく,Google chromeを推奨する.なぜなら,SafariはApple独自のエンジンで動いており,不具合が生じる可能性があるからである.

\url{http://localhost:8080/}

このときに,トップページが表示され,各システムへのリンクが機能していれば,正常に稼働している.

\subsection{停止方法}
\ref{kido}にてサーバーを起動したターミナル上で,ControlキーとCキーを同時に入力することで,サーバプロセスを停止することができる.

\newpage
\section{運用・管理上の注意点}
\subsection{データの保存について}
本システムは簡易データベースとして,プログラム内の変数にデータを保存する仕組みをとっている.このため,サーバを停止したり再起動したりすると,サーバ運用中に追加・編集・削除されたデータは全てリセットされ,初期状態に戻る.

\subsection{ポート番号の競合について}
本システムは,ポート8080番を使用する.起動時にEADDRINUSE等のエラーが出る場合は,既に他のアプリケーションが8080番ポートを使用している可能性がある.その場合,競合しているアプリケーションを停止するか,IDapp.jsの最終行に記載されているポート番号の変更が必要となる.

\subsection{ログの確認}
システムの稼働中,データの追加や編集,削除が行われると,ターミナル上にログが出力される.システムの動作状況を確認する際には,サーバを起動したターミナルの出力を参照することで確認が可能である.